\section{Problem definiton}
\subsection{Problem}
IoT is a big field with prospect, probably adding of tens of trillions of dollars to GDP within ten years. The problem which I would like to solve is to show possibilities of connected vehicles providing more safety to the roads, lowering industry expenses on vehicle park, and moreover better services to the end user.
\subsection{Plan}
To tackle this problem first we need to develop embedded device which will enable us to collect data from vehicles. Data can be collected from vehicle's various sources. Our main source of information will be OBD-II service port. OBD-II is industry standard port for gathering information about vehicle state from main ECU, which is mandatory in all gasoline(petrol) vehicles since 2001 and all diesel vegicles since 2003 in European Union. When we have collected data on our embedded device, we need to connect it to the Internet. where we will be able to transfer data to our server. After processing we will create API to be able to build applications on top of our data, which can provide various services to the end user.\\
%Solution
\subsection{Output}
To show capabilities of connected car, I have choosen this two applications which will be presented in this thesis. First applicaton will try to increase security of people traveling by vehicle with accident detection and prevention.Accident prevention will alert driver that he is passing through dangerous zone. If accident happens, connected vehicle will send message to the emergency hotline. As for second application, it will be fuel statistic/marketing based application. App will create map of cars having low fuel, ergo there is high chance what they would like to stop at gas station for refill. Owners of that vehicles then could be proposed with marketing option to come refill their tank at advertised station. 

%Content: IoT
\section{Internet of Things} 
\subsection{General Information}
The Internet of Things (IoT) is a global infrastructure for further informatization of society, enabling advanced services by interconnecting things based on available technologies. Through identification, data capture, communication and processing capabilities, the IoT makes full use of things to offer services to all kinds of applications. Whilst IoT is a hot topic in the industry it is not a new concept. It was initially put forward by Mark Weiser in the early 1990s. This concept is opening up huge opportunities for both the society and individuals. However, it also involves risks and undoubtedly represents an immense technical and social challenge.\
\subsection{Concept}
\subsection{Benefits}
\subsection{Why is it important?}
In my thesis I will connect a thing(car) to the internet so following principles which was set by IoT field will allow me to build better and more secure solution.
%Content: Big Data
\section{Big Data}
\subsection{General Information}
The rise of digital and mobile communication has made the world become more connected, networked, and traceable and has typically lead to the availability of such large scale data sets that the traditional data processing applications are insufficient. As the result new field trying to deal with this problem, have been created which scientists and computer engineers have coined `Big Data'.
\subsection{Concept}
\subsection{Benefits}
\subsection{Why is it important?}
When we will have connected car, we will need to collect information, e.g. data. With Big Data approach our solution will be scaleable without need for adaptation if a very big number of devices would be connected.
\section{Single-board computer}
\subsection{General Information}
A single-board computer(SBC) is complete functional computer built on single circuit board. It has microprocessor, memory, input/output and other features depending on the model and manufacturer. Single-board computers are used for educational purposes, embedded solutions and development research/systems.
\subsection{Why is it important?}
Since we will be building embedded device which will allow us to gather data and send them to the Internet, it is necessary to choose a single-board computer which will fit fulfill our needs.

\section{OBD-II port}
\subsection{General Information}
OBD II is an acronym for On-Board Diagnostics II, the second generation of on-board self-diagnostic equipment. On-board diagnostic capabilities are incorporated into the hardware and software of a vehicle's on-board computer to monitor virtually every component in the car. Each component is checked by a diagnostic routine to verify that it is functioning properly. If a problem or malfunction is detected, the OBD II system will alert the driver that something is wrong. The system will also store important information about any detected malfunction so that a repair technician can accurately find and fix the problem.
\subsection{}

\section{Accelerometer}
\subsection{General Information}
%problem




