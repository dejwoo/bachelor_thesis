%Content: IoT
\section{Internet of Things} 
\subsection{General Information}
The Internet of Things (IoT) is a global infrastructure for further informatization of society, enabling advanced services by interconnecting things based on available technologies. Through identification, data capture, communication and processing capabilities, the IoT makes full use of things to offer services to all kinds of applications. Whilst IoT is a hot topic in the industry it is not a new concept. It was initially put forward by Mark Weiser in the early 1990s. This concept is opening up huge opportunities for both the society and individuals. However, it also involves risks and undoubtedly represents an immense technical and social challenge.\cite{iot}\

``Things'', in the Interten of Things refer to a various devices. From hardware level they are all designed to do different things, collect data from various enviroments and sources. Howewer, in software aspect they behave more generally, in simplistic from its a thing which can report data and act upon them. Objectification is important, because then you can combine many things to work together and communicate between them. Current market example could be the smart thermostat systems combinating washer/dryers that use Wi-Fi for remote access.

\subsection{Benefits}
IoT is generating a lot of interest in a wide range of industries. Here are a few examples of some significant
early adopters:\\
• In the healthcare field, medical device manufacturer Varian Medical Systems is seeing a 50 percent reduction
in mean time to repair their connected devices.\cite{varian_reduce} With IoT, Varian reduced customer service costs by
\$2,000 for each problem resolved remotely, with 20 percent fewer technician dispatches worldwide.\\
• Tire maker Pirelli is using IoT to gain valuable insights about the performance of its products in nearreal
time. The company is using an analytics platform to manage the huge amounts of data gathered
directly from sensors embedded in the tires in its Cyber Tyre range. The system allows the pressure,
temperature, and mileage of each tire to be monitored remotely. By keeping these factors in range, fleet
managers can have a significant impact on fuel economy and safety. In a trial covering nearly 10 million
miles, Cyber Tyres saved the equivalent of \$1,500 per truck per year.\\
• Ford Motor Company’s Connected Car Dashboards program collects and analyzes data from vehicles
in order to gain insights about driving patterns and vehicle performance. The data is analyzed and
then visualized graphically using a big data platform. Among the goals are better vehicle design and
improved safety for occupants.\\
• In the public service sector, the Boston police department operates a “real-time crime center” that
receives dozens of feeds from street cameras and other sensors around the city. The resulting data gives
researchers the ability to analyze and match videos from incidents to help identify suspects, mobilize
resources, and even map evacuation routes during emergencies\\

IoT has the potential to transform the way companies make products, track goods and assets in the supply
chain, monitor the performance of systems in the field, provide security for employees and facilities,
and provide services to customers. Clearly, it’s enabling transformation in both the private and public
sectors. “I firmly believe that there is not a single industry that won’t benefit from IoT,” Turner said.
IoT is also changing the way businesses impact society and the environment. “Overall, the world needs better
and more sustainable ways to live,” said Stephen Miles, research affiliate at the Center for Biomedical
Innovation at the Massachusetts Institute of Technology. “To accomplish this, companies need better,
more holistic models that capture a complete picture of what is happening so that they can better access
and optimize these systems.”

\subsection{Why is it important?}
In my thesis I will connect a thing(car) to the internet so following principles which was set by IoT field will allow me to build better and more secure solution\cite{IoT_standart}.
%Content: Big Data
\section{Big Data}
\subsection{General Information}
The rise of digital and mobile communication has made the world become more connected, networked, and traceable and has typically lead to the availability of such large scale data sets that the traditional data processing applications are insufficient. As the result new field trying to deal with this problem, have been created which scientists and computer engineers have coined `Big Data'.
\subsection{Concept}
\subsection{Benefits}
\subsection{Why is it important?}
When we will have connected car, we will need to collect information, e.g. data. With Big Data approach our solution will be scaleable without need for adaptation if a very big number of devices would be connected.
\section{Single-board computer}
\subsection{General Information}
A single-board computer(SBC) is complete functional computer built on single circuit board. It has microprocessor, memory, input/output and other features depending on the model and manufacturer. Single-board computers are used for educational purposes, embedded solutions and development research/systems.
\subsection{Why is it important?}
Since we will be building embedded device which will allow us to gather data and send them to the Internet, it is necessary to choose a single-board computer which will fit fulfill our needs.

\section{OBD-II port}
\subsection{General Information}
OBD II is an acronym for On-Board Diagnostics II, the second generation of on-board self-diagnostic equipment. On-board diagnostic capabilities are incorporated into the hardware and software of a vehicle's on-board computer to monitor virtually every component in the car. Each component is checked by a diagnostic routine to verify that it is functioning properly. If a problem or malfunction is detected, the OBD II system will alert the driver that something is wrong. The system will also store important information about any detected malfunction so that a repair technician can accurately find and fix the problem.
\subsection{Concept}
The OBD-II standard specifies the type of the connector and its pinout, the messaging format and electrical signalling protocols available. OBD-II also provides a list of vehicle accessible parameters for monitoring tohether with how to encode the data for each. One of the pin in the connector provides power to the connected unit, so it could run from the vehicle battery, which simplify use of scantools, because you do not need auxiliary power. On the other hand, sometimes auxiliary power is needed, because of car malfunction which could shutdown the scan tool and lead to loss of diagnostic data. Finally, the OBD-II standard provides an extensible list of DTCs(Diagnostic trouble codes). As a result of this standardization, a single device can query the on-board computer(s) in any vehicle. OBD-II standardization was prompted by emissions requirements, and though only emission-related codes and data are required to be transmitted through it, most manufacturers have made the OBD-II Data Link Connector the only one in the vehicle through which all systems are diagnosed and programmed. OBD-II Diagnostic Trouble Codes are 4-digit, preceded by a letter: P for engine and transmission (powertrain), B for body, C for chassis, and U for network. \cite{obdiso}
\subsection{Why is it important?}
One of the main sources of information in my thesis will be data from OBD-II diagnostic port. This data will be accessible later in the database and we can preform logical deduction on them.
\section{Accelerometer}
\subsection{General Information}
An accelerometer measures proper acceleration which is the acceleration it experiences relative to freefall, and is the acceleration that is felt by people and objects. Put another way, at any point in spacetime the equivalence principle guarantees the existence of a local inertial frame, and an accelerometer measures the acceleration relative to that frame.\cite{einstein_rel}As a consequence an accelerometer at rest relative to the Earth's surface will indicate approximately 1 g upwards, because any point on the earth's surface is accelerating upwards relative to a local inertial frame. To obtain the acceleration due to motion with respect to the earth, this "gravity offset" should be subtracted.\\
The reason for the appearance of a gravitational offset is Einstein's equivalence principle\cite{equivalence}, which states that the effects of gravity on an object are indistinguishable from acceleration of the reference frame. When held fixed in a gravitational field by, for example, applying a ground reaction force or an equivalent upward thrust, the reference frame for an accelerometer (its own casing) accelerates upwards with respect to a free-falling reference frame. The effect of this reference frame acceleration is indistinguishable from any other acceleration experienced by the instrument.
An accelerometer will read zero during free fall. This includes use in a spaceship orbiting earth, but not a (non-free) fall with air resistance where drag forces reduce the acceleration until terminal velocity is reached, at which point the device would once again indicate 1 g acceleration upwards.\\
\subsection{Why is it important?}
For accident detection we will use accelerometer to determine if accident occured\cite{accident}. Application will proactively monitor the occurrence of accident using accelerometer sensor. When the
accelerometer sensor values exceedes the threshold value, system can act upon it.




