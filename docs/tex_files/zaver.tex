In this chapter we summarize what was achieved in our solution and compare it to the requirements and then suggests what can be done in future work of this bachelor thesis.
\section{Result} % (fold)
\label{sec:result}
We provided embedded device which is running Arch Linux operating system, all peripheral devices were successfully connected to the device and are operating. Accelerometed data are recored through \gls{i2c} protocol, cellular modem is capable of connecting to the internet and sending SMS messages, Wi-Fi extension dongle is automaticaly used to create Software Access Point to which anybody with password can connect. \gls{obd} port is interconnected with device throug ELM-USB cable and is properly configured through our application to connect to the car.\\\\
Application was developed which allows collecting of data from multiple sources. Application is configurable, modular and modules are interchangeable. Internal communication between modules is designes so modules can send and accpet data. In routes modules are selectable by types, which allows to define large dataflows with simple routing. Modules are provided for gathering sensoric data from accelerometer, OBD-II port of vehicle, NMEA data from GPS sensor and time from OS. Provided modules for sending data are for sending SMS messages, sending messages over AMQP protocol triggering IFTTT events, storing data into Redis database and displaying data to the standard output of Node.js application. By combing this low-level modules and adding logic high-level modules are built. Modules which we developed are for reporting accidents with SMS messages, advising drivers when to shift gears based on Revolutions per Minute(RPM) of the car, sending data in bulk mode over cellular network instead of constant stream. Routes for correct work of the application are provided.\\\\
Web application is accessible through wireless access point and is used for configuring modules and displaying real-time data from modules. Applicaton uopn correct instalation on embedded device works and is capable of monitoring various inputs. This information is collected, distributed and made available for differend services.\\\\
If we compare requirements and results we can see we fullfiled requirements of this bachelor thesis.
% section result (end)
\section{Future work} % (fold)
\label{sec:future_work}
Main goal of future work is to make available for everyone our Data-Logger framework, create installation module in the npm.js module directory. After Data-Logger is available start producing application modules for various sensors, devices and services based on popular demand. Create guide or wiki pages for better understanding of how Data-Logger works and to attract additional developers to extend it.\\\\
In modules there are even more possibilities what could be done so we name a few. RabbitMQ module should be implemented with SSL/TLS encryption to ensure data privacy when sending messages to server on the Internet. Implement additional modules to gather data from all sensors on Adafruit 10DOF breakout board as this thesis implements module for accelerometer sensor. Design more high-level modules for buisness logic. Research how to make embedded device smaller and cheaper which would allow manufacturing on larger scale.
% section future_work (end)