\documentclass[12pt,a4paper]{bachelor}

\usepackage[english]{babel}
%\usepackage{ucs}
\usepackage{amsmath}
\usepackage{amsfonts} %amsthm ?
\usepackage{amssymb}
\usepackage[T1]{fontenc}
\usepackage[utf8]{inputenc}
\usepackage[top=2.5cm, bottom=2.5cm, right=2cm, left=3.5cm]{geometry}  %okraje
\usepackage[toc]{appendix}
\usepackage{graphicx}
\usepackage{times}
\usepackage{listings}
\usepackage{color}
\usepackage[bf,belowskip=-18pt,aboveskip=4pt]{caption}
\usepackage[natbib=true, style=fmph]{biblatex}
\bibliography{bibliography}
\usepackage{booktabs}
\usepackage{tabularx}
\usepackage{float}
\usepackage[compact]{titlesec}
\titlespacing{\section}{0pt}{28pt}{24pt}
\titlespacing{\subsection}{0pt}{25pt}{15pt} %left top down
\setcounter{secnumdepth}{5}
\setcounter{tocdepth}{5}
\usepackage{csquotes}

%toto sposobi ze slova sa nebudu na konci riadkov rozdelovat pomlckami
\tolerance=100000000
\hyphenpenalty 10000000
\exhyphenpenalty 10000000

%zadefinovane premenne ktore sa potom pouzivaji na patricnych miestach
\def\skola{Comenius University in Bratislava}
\def\fakulta{Faculty of Mathematics, Physics and Informatics}
\def\nazov{Internet of Things in automotive industry}
\def\autor{Dávid Kőszeghy}


\usepackage[pdftex,bookmarks=true]{hyperref}
\hypersetup{
colorlinks=true,
linkcolor=black,
urlcolor=black,
pdftitle={\nazov},
pdfauthor={\autor},
pdfsubject={\fakulta}
}

\title{Comenius University in Bratislava
Faculty of Mathematics, Physics and Informatics}
\author{\autor}

\hypersetup{citecolor=black}

\setlength{\parindent}{0pt} 
\definecolor{light-gray}{gray}{0.93}
\definecolor{darkgray}{rgb}{0.2,0.2,0.2}
\definecolor{brown}{rgb}{0.3,0.15,0.0}

%formatovanie zdrojoveho kodu - pre viac info a moznosti pouzi google
\lstset{ %
basicstyle=\footnotesize\ttfamily, 
backgroundcolor=\color{light-gray},  
language=PHP,
breaklines=true,
%columns=flexible,  
showspaces=false, % show spaces adding particular underscores
showstringspaces=false, % underline spaces within strings
escapeinside={\%*}{*)}, % if you want to add a comment within your code
commentstyle=\color{darkgray},
stringstyle=\color{brown},
}
%nastavenie klucovych slov ktore maju byt inak zvyraznene
\lstset{
morekeywords={try,class,extends,__construct,var, construct,this,parent,catch,new,protected,private,public,function,return}, 
keywordstyle=\color{black}\bfseries
}
% Load the package
\usepackage{glossaries}
% Generate the glossary
\makeglossaries


%koniec nastaveni a tu uz zacina obsah
\begin{document}
\newacronym{obd}{OBD-II}{On-Board Diagnostics port}
\newacronym{iot}{IoT}{Internet of Things}
\newacronym{sbc}{SBC}{Single-board computer}
\newacronym{gpio}{GPIO}{General-purpose input/output}
\newacronym{i2c}{I${}^2$C}{Inter-Integrated Circuit}
\newacronym{spi}{SPI}{Serial Peripheral Interface}
\newacronym{lan}{LAN}{Local Area Network}
\newacronym{ecu}{ECU}{Electronic Control Unit}
\newacronym{can}{CAN}{Controller Area Network}
\newacronym{epa}{EPA}{Environmental Protection Agency}
\newacronym{sae}{SAE}{Society of Automotive Engineers}
\newacronym{oem}{OEM}{Original equipment manufacturer}
\newacronym{pids}{PIDs}{Parameter IDs}
\newacronym{v8}{V8}{V8 JavaScript Engine}
\glsunsetall
\glsresetall
\glsunset{lan}


\input firstpages.tex
\input abstract.tex

\tableofcontents
\newpage

\chapter*{Introduction}
\input 00_introduction.tex
\addcontentsline{toc}{chapter}{Introduction}
\chapter{Overview}
\input 01_overview.tex

\chapter{Specification}
\input 02_specification.tex

\chapter{Solution}
\input 03_solution.tex

\chapter{Implementation}
\input 04_implementation.tex

\chapter{Conclusion}
\input zaver.tex

\printglossaries
\printbibliography
\addcontentsline{toc}{chapter}{References}  %pridanie nadpisu zdrojov k obsahu

\begin{appendices}

\input prilohy.tex
\end{appendices}
\label{totalpages}
\end{document}
